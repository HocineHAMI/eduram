\documentclass[11pt,a4paper,titlepage]{article} 
\usepackage[utf8]{inputenc} 
\usepackage[french]{babel} 
\usepackage[T1]{fontenc} 
\usepackage{amsmath} 
\usepackage{amsfonts} 
\usepackage{amssymb} 
\usepackage{graphicx} 
\usepackage[final]{pdfpages} 
\usepackage[toc,page]{appendix} 
\usepackage[top=2.5cm,bottom=2.5cm,right=2.5cm,left=2.5cm]{geometry} 
\author{GLANGINE Geoffrey, SENE Victor } 
\title{La voiture autonome} 
\graphicspath{{Images/}} 
\usepackage{fancyhdr} 
\pagestyle{fancy} 
\usepackage{lastpage} 
\renewcommand
\headrulewidth{1pt} 
\fancyhead[L]{Éduram} 
\fancyhead[C]{GLANGINE Geoffrey, SENE Victor } 
\fancyhead[R]{LO43 UTBM} 
\renewcommand
\footrulewidth{1pt} 
\fancyfoot[C]{\textbf{Page \thepage}} 
\fancyfoot[L]{Automne 2015}

\newcommand{\HRule}{\rule{\linewidth}{0.5mm}}

\begin{document}

\begin{titlepage}
\begin{center}

% Upper part of the page. The '~' is needed because \\
% only works if a paragraph has started.
%\includegraphics[width=0.15\textwidth]{./logo}~\\[1cm]

\textsc{\LARGE Université de technologie Belfort-Montbéliard}\\[1.5cm]

\textsc{\Large Éduram (Adaptation du jeu Pandémie)}\\[0.5cm]

% Title
\HRule \\[0.4cm]
{ \huge \bfseries LO43 Bases Fondamentales de la programmation orientée objet\\[0.4cm] }

\HRule \\[1.5cm]

% Author and supervisor
\begin{minipage}{0.4\textwidth}
\begin{flushleft} \large
\emph{Auteur:}\\
\textsc{GLANGINE} Geoffrey\\
\textsc{SENE} Victor\\
\end{flushleft}
\end{minipage}
\begin{minipage}{0.4\textwidth}
\begin{flushright} \large
\emph{Professeur:} \\
Gechter \textsc{Franck}
\end{flushright}
\end{minipage}
\vfill

% Bottom of the page
{\large \today}

\end{center}
\end{titlepage}
\tableofcontents
\pagebreak

\chapter{Qu'est-ce que Eduram ?}

\section{Un jeu au règles complexes}

Eduram est un jeu basé sur le célèbre jeu Pandémie (Pandémic en version originale). Le principe du jeu est d'essayer d'empêcher des virus de se propager dans les villes du monde entier. C'est un jeu de coopération ou les 
joueurs jouent tous dans la même équipe contre le jeu. Les règles du jeux sont plutôt simple, chaque joueur possède des cartes avec des noms de ville et des couleurs car la carte est séparée en plusieurs continents
et chaque continent possède une couleur. Les joueurs peuvent effectuer 4 actions par tour (se déplacer, créer un vaccin contre 5 cartes de la même couleur, enlever un virus sur la ville ou ils se trouvent actuellement, donner une carte à un ami qui se trouve sur la même ville qu'eux ou encore effectuer une action liée à leur spécialité ).
En effet chaque joueur possède un rôle qui lui permet de faire des actions différente des autres joueurs ( le scientifique peut créer un vaccin pour seulement 4 cartes ).
Le jeu est perdu si il y a eu plus de 8 d'éclosions, si il n'y a plus de cartes dans la pioche des joueurs, ou si on ne peux plus placer de cubes de virus sur la carte. La partie est gagné lorsque tous les vaccins ont été créés.

\section{Nos choix d'adaptation du jeu}

Nous avons décidé de refaire une version utbohémiene du jeu. Nous avons donc choisi du représenter le réseau de l'UTBM qui se fait infecter par plusieurs virus de type informatique. En effet, nous avons choisi de représenter sur notre carte les bâtiments principaux de l'UTBM avec des salles. Un joueur situé sur une salle peut se déplacer dans les salles adjacentes pour un coût de 1 action.
\\Nous avons aussi décidé de remplacer les cartes de joueur par des mots de passes pour coller un peu mieux avec le secteur de l'informatique. Donc les joueurs pourront s'échanger des mots de passe comme les cartes dans le 
jeu original. Les joueurs pourront aussi créer des anti-virus pour supprimer des virus. La partie sera donc gagnée lorsque tous les anti-virus seront créés. Les conditions de défaites sont les mêmes que dans le pandémie original.
\subsection{Implantation des rôles de joueurs}
Nous avons décidé dans un premier temps de faire seulement 3 rôles : 
\begin{itemize}
 \item L'administrateur système : il peut se déplacer de bâtiments en bâtiments plutôt que de salles en salles.
 \item Le technicien : il peut enlever tous les virus de la salle ou il se trouve pour le coût d'une seule action.
 \item L'enseignant-chercheur : Peut créer un anti-virus avec seulement 4 mots de passes.
\end{itemize}
\subsection{Modes de déplacement}
Le jeu pandémie présentait plusieurs modes de déplacement (voiturier, vol nolisé et vol direct)
Nous avons décidé de remplacer les moyens de transport par des types de communication réseau.
\begin{itemize}
 \item SSH : représente le vol nolisé (on défausse le mot de passe de la salle où l'on se trouve pour se rendre dans n'importe quelle autre salle)
 \item Telnet : représente le vol direct (on défausse un mot de passe pour se rendre dans la salle associée)
 \item Console : représente le déplacement voiturier (d'une salle vers une salle adjacente à celle ci)
\end{itemize}







\end{document}